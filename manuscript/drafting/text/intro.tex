\documentclass[main.tex]{subfiles}


\begin{document}

\section{Introduction}

Chemical probing coupled with mutational profiling has enabled genome-wide analyses of RNA structure \textit{in vivo}~\cite{Spitale2023}.
All such methods use a chemical to label unpaired/flexible nucleotides in RNA, reverse transcription to encode labels as mutations in cDNA, and next-generation sequencing to detect the mutations.
Dimethyl sulfate mutational profiling with sequencing (DMS-MaPseq)~\cite{Zubradt2016} uses DMS to methylate unpaired adenine (A) and cytosine (C) bases at the Watson-Crick face and thermostable group II intron reverse transcriptase (TGIRT-III) or Induro~RT to encode methylations as mutations~\cite{RomeroAgosto2024}.
The recently developed method 1-ethyl-3-(3-dimethylaminopropyl) carbodiimide mutational profiling (ETC-MaP)~\cite{Douds2024} uses ETC to modify unpaired guanine (G) and uracil (U) bases -- complementing DMS -- and TGIRT-III for reverse transcription.
And selective 2'-hydroxyl acylation analyzed by primer extension with mutational profiling (SHAPE-MaP)~\cite{Siegfried2014} uses 2'-hydroxyl-selective ``SHAPE" reagents to create covalent adducts at conformationally flexible nucleotides, which are encoded with SuperScript~II~RT.

Mutational profiling can also reveal aspects of RNA biology beyond structure.
Several endogenous RNA modifications can be detected by reverse transcription, including inosine~\cite{Okada2019}, m\textsuperscript{1}A~\cite{Li2017}, and m\textsuperscript{5}C~\cite{Schaefer2008}.
RNA-protein interactions have also been probed using mutational profiling with SHAPE reagents~\cite{Smola2015} and NHS-diazirine~\cite{Weidmann2021}.
Collectively, these methods can investigate RNA structures, modifications, and interactions with proteins -- and the connections between these fundamental components of RNA biology.

A key feature of mutational profiling is that it can be used to detect alternative RNA structures.
Many RNA molecules fold into multiple structures, collectively called an RNA structure ensemble~\cite{Spitale2023}.
By default, mutational profiling data reflect the average across all structures in the ensemble.
But individual structures can be resolved by clustering the sequencing reads, which is possible because each read came from one RNA molecule in one structural state and carries a set of mutations specific to that structure.
Multiple pieces of software have been developed to resolve RNA structure ensembles by clustering mutational profiling reads, including DREEM~\cite{Tomezsko2020}, DRACO~\cite{Morandi2021}, and DanceMapper~\cite{Olson2022}.

However, every piece of existing software has limitations.
Most significant is the inability to correct inherent biases in the data that can yield spurious structures when clustering reads.
DREEM~\cite{Tomezsko2020} and DanceMapper~\cite{Olson2022} both cluster reads using the expectation-maximization algorithm with a Bernoulli mixture model that assumes mutations occur independently.
However, in DMS-MaPseq data, mutations less than 4~nt apart are underrepresented~\cite{Tomezsko2020}, which violates the assumption of independence.
Although DREEM does introduce a correction for this bias, it works only when all reads span the full sequence being clustered, limiting its utility for long transcripts and sequencing libraries generated by random fragmentation/priming.
Finding alternative structures across long transcripts also poses a challenge, so far addressed only by DRACO~\cite{Morandi2021} using a sliding window approach that critically depends on selecting the proper window length.
No existing software can detect PCR jackpotting~\cite{Marotz2019}, a common type of bias that can also fool clustering algorithms into detecting false alternative structures.
Additional limitations are in usability: users must install software and dependencies individually, process input files one at a time, track every file through the workflow, and record parameters manually.

Here, we introduce SEISMIC-RNA, software for analyzing RNA mutational profiling data and resolving RNA structure ensembles.
SEISMIC-RNA addresses the critical limitations of previous software.
Users can run the full workflow on an unlimited number of files in parallel with one concise command (\verb|seismic wf|), and SEISMIC-RNA will track every file automatically and write reports summarizing the parameters and results.
It corrects the bias against closely spaced mutations in both amplicons and randomly fragmented/primed sequencing libraries, enabling accurate analysis of transcriptome-wide mutational profiling data.
SEISMIC-RNA introduces a new method to detect alternative structures across long transcripts that does not rely on a fixed window size, making it more robust than previous approaches.
It also features a new algorithm to detect PCR jackpotting and other stringent filters to verify that clusters are authentic and reproducible.
And it offers an array of tools for analyzing data (e.g. correlations and ROC curves) and simulating mutational profiling experiments from scratch, as well as a Python API.
Benchmarking results indicated that SEISMIC-RNA is the fastest, most accurate software for resolving RNA structure ensembles.
SEISMIC-RNA and all its dependencies can be installed with Conda (\verb|conda install seismic-rna|), and its source code is available from GitHub (\url{https://github.com/rouskinlab/seismic-rna}).

\end{document}
